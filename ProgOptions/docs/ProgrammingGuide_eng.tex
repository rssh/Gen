
\documentclass[10pt]{article}
%\usepackage{tex4ht}
\usepackage[OT1,T2A]{fontenc}
\usepackage{graphicx}
\usepackage{verbatim}

\title{ ProgOptions: Programmers Guide  
       \newline
       \small{DocumentId:GradSoft-PR-e-26.09.2000-v1.0.3} 
      }

% $Id: ProgrammingGuide_eng.tex,v 1.20 2002-04-11 15:06:57 kav Exp $

\begin{document}

\maketitle{}

\tableofcontents

\section{ Introduction }

 ProgOptions is a component for handling of program command string options.

 Using ProgOptions it is possible to simply implement following facilities:
 \begin{enumerate}
    \item ascertaining the presence in the command line of some (previously determined) options
    \item getting option arguments ( - note the ProgOptions allows to use a few syntax style,
          for example, the
\begin{verbatim}
                  --a Oh 
\end{verbatim}
          syntax for "a" option with "Oh" argument is valid as well as the
\begin{verbatim}
                  --a="Say Oh"
\end{verbatim}
          syntax for the same option with another "Say Oh" argument;
    \item as soon as the presence of sertain option is ascertained, invoking the function connected.
 \end{enumerate}


 This document is a non-formal description of package, the full specification is present in
 API reference. \\
 (\verb|www.gradsoft.kiev.ua/common/ToolBox/ProgOptions/API/index.html|).

\section{ The basic way of using }
\label{base}

 The "basic" way of ProgOption usage is suitable for the case of command line accessble through
 the pair (argc,argv) with personnels determined as (int) and (char**) obtained through the "main"
 function arguments.
 If this case is present, obtaining of ProgOptions functionality needs following steps:

 \begin{itemize}
   \item Construct a new instance of ProgOptions keeping in mind the ProgOptions::ProgOptions
   	     is defined as follows:

         ProgOptions::ProgOptions(const char* optPrefix="--",
                                  const char* pkgPrefix="",
                                  bool allowUnknownOptions=false);

         where
   \begin{enumerate}
   \item optPrefix is the prefix by means of which end user must marking out options recognizable
         (i.e. previously described)
   \item pkgPrefix is a substring for marking out special options "help" and "config" 
         (see following for them)
   \item allowUnknownOptions is boolean adjusting precedence rule for the case of don't recognizable
         option(s) appear. The "true" is mind that the presence of don't recognizable options will
         be ignored while the "false" is mind that the parsing of command line will be interrupted
         and error message will be dispatched to stdandart error stream.
   \end{enumerate}

   \item Describe some options will be recognized and it's connected actions using method ProgOptions::put
         and set additional help messages using method ProgOptions::setAdditionalHelp

   \item Call ProgOptions::parse for parsing argv. After this step the command line divided to
         argv will be parsed, and options recognized will be placed to internal structures
         of ProgOptions, and all callback functions will be called.

   \item After this with usage of \verb|ProgOptions::is_set("option")| you can check option setting.
         For options with argument you can read value of argument by using 
         ProgOptions::argument("option")
 \end{itemize}

Example:

\begin{verbatim}

#include <GradSoft/ProgOptions.h>  // first step

GradSoft::ProgOptions options;   // construct a new instance of ProgOptions: 

void init()
{
 options.put("qqq", "qqq option", false );           // discribe the option "qqq"
                                                     // which not need arguments
 options.put("zzz", "option with argument", true );  // discribe the option "zzz"
                                                     // which need an argument

 options.setAdditionalHelp(true,                     // 
   "This program illustate usage of GradSoft library"// set additional help messages
 );                                                  //  
 options.setAdditionalHelp(false,                    // 
   "and this is shown at the end of usage screen"    // 
 );                                                  //
};

int main(int argc, char** argv)                      
{
 init();                                       

 if (!options.parse(argc,argv)) return 1;      // parse options

 if (options.is_set("qqq")) {                  // find out, is the option seted
   cout << "qqq is set" << endl;
 }
                                               // find out, is the option seted
                                               // and get his argument if yes
 if (options.is_set("zzz")) {
   cout << "zzz is set with argument:" << options.argument("zzz") << endl;
 }
 return 0;
}

\end{verbatim}

Now run this program:

\begin{verbatim}
./a.out --qqq --zzz zz-arg
qqq is set
zzz is set with argument:zz-arg
\end{verbatim}

And now run with option not described:

\begin{verbatim}
./a.out --Go
a:Go:unknown option
\end{verbatim}


\subsection{ Special options help and config}


  In addition to set of options, described by programmer, ProgOptions define
two specially acting options: \verb|--<pkgPrefix>help| and \verb|--<pkg>config|.
The result of calling program with option \verb|--<pkgPrefix>help| is passing
to stdandart error stream (cerr) a standard help message (which is the list of options
with descriptions inputted on the step of using ProgOptions::put) added by additional
help messages specified via ProgOptions::setAdditionalHelp.
The result of calling program with \verb|--<pkg>config fname| is parsing options described
in the file \verb|<fname>| (see next section \ref{Additional resources} for more details).

Example:

	Start program mentioned with option \verb|--help:|:

\begin{verbatim}
./a.out --help
This program illustate usage of ProgOptions library
  --qqq 		qqq option
  --zzz <argument> 	option with argument
and this is shown at the end of usage screen
\end{verbatim}

\noindent {\bf Note:}
Using \verb|--config <filename>| leads to some troubles for us
when some part of options must be passed into "external" software,
let us, into function ORB\_init().
The point is that the ProgOptions::parse(argc,argv) method merge all options,
given in argv vector and extracted from the file 
in especial argument vector being used then, and do not change starting argv.
Therefore, if you call ORB\_init() with argc and argv being parameters of main(),
options from the file will be lost.
Therefore, if you want to pass command-line options from the file into ORB\_init() really,
you must:
\begin{enumerate}
\item parse initial argument vector using "parse(argc,argv)";
\item pass to ORB\_init() an argument vector and its length
      having been created during parsing.
\end{enumerate}
At the same time, since an external software can remove options recognized by itself,
it's the most correctly to give it a copy of argument vector created, and its length, 
in which purpose wrapper-class ProgOptions::ArgsHolder can be used:
\begin{verbatim}
ProgOptions options("--","",true);
if (!options.parse(argc,argv)) return 1;

ProgOptions::ArgsHolder argsHolder;
argsHolder.takeArgv(options);

CORBA::ORB_var orb = CORBA::ORB_init(argsHolder.argc,argsHolder.argv);
\end{verbatim}
It is handly, because new argv created via argsHolder::takeArgv(const ProgOptions\&)
will be deleted automatically when destructor of argsHolder will be called.

\noindent {\bf Note:} to date, syntax \verb|--config=<filename>| do not possible
for special option \verb|--config|. Use \verb|--config=<filename>| instead. \\

You can overload special option \verb|--help| and \verb|--config|
using method ProgOption::put. Overloaded option has no feature and default action.


\section{ callback functions  }

 Yet one way of handling options handling: callback functions. You can specify
 such function in parameter |callback| of method \verb|ProgOptions::put| and
 it will be called during parsing appropriate option.

 To illustrate this, let's add to our program next piece of code:

\begin{verbatim}

bool zz1Callback(const GradSoft::ProgOptions* options, const char* argument,
                 void* )
{
 cout << "zz1:callback called with argument " << 
            ((!argument) ? "NULL" : argument) << endl;
 return true;
}

void init()
{
 .....
 options.put("zz1", "option with argument and callback", true, zz1Callback );
};

\end{verbatim}

 Now run program with this option:

\begin{verbatim}
/a.out --zz1 xx
zz1:callback called with argument xx
\end{verbatim}

 Callback function is called during each occurence of option in command string,
 so you can organize handling of few identical options.

\begin{verbatim}
/a.out --zz1 xx --zz1 yy
zz1:callback called with argument xx
zz1:callback called with argument yy
\end{verbatim}


\section{ Additional resources }

\subsection{ Settiong of ProgOptions properties }

 \begin{itemize}
  \item optPrefix, pkgPrefix (const char*)-- this properties control syntax, for options, handled by ProgOptions: it looks for options in form: \\
    \verb|<optPrefix><pkgPrefix><optionName>|.
   \begin{itemize}
     \item Default value: \verb|--| and empty string accordinally 
     \item Way of setting: parameters of constructor.
   \end{itemize}
  \item allowUnknownOptions (bool) -- if this property is set to true,
  than during parsing ProgOptions ignore elements of command string, which
   do not follow ProgOptions syntaxm or options, which was not set via
   ProgOptions::put. If this property is set to false, than ProgOptions::parse
   break parsing and return false on first occurence of such element.
   \begin{itemize}
     \item Default value: false
     \item Way of setting: 
     \begin{itemize}
       \item via parameter of constructor 
       \item \verb|ProgOptions::setAllowUnknownOptions(bool);|
     \end{itemize}
     \item reading: \verb|bool ProgOptions::getAllowUnknownOptions()|
   \end{itemize}
 \end{itemize}



\label{Additional resources}

\subsection{ Using of the config file }

  As it will be mentioned above, if you use ProgOptions::parse in you program,
it will lead to options described in the file \verb|fname| will be parsed
if the end user of program apply \verb|--config fname| in the command line.

  The additional possibility is that:  you may read options from file and parse them
independently of end user choice using special method
ProgOptions::parseFile(const char* configFname, const char* executable="unspecified").
For example, calling

\begin{verbatim}

        options.parseFile("D:\Demo\config.ini");

\end{verbatim}
(where options is instance of ProgOptions)
means that the text in the \verb|D:\Demo\config.ini| will be readed, parts of text
recognized as "options" will be extracted and then interpreted as well as
these options be in command line. (Second parameter of this method constitutes the name
of the program asking for options and is optional: you may set it
in the case of call of ProgOptions::parse not precede the call of ProgOptions::parseFile.)

  The next mean allow you to combine all option parsed by ProgOptions::parse,
ProgOptions::parseFile and ProgOptions::parseString (see following for the latter)
and save these into the file specified using method
\verb|ProgOptions::saveToFile|.
For example, using 

\begin{verbatim}

        ProgOptions options;
        ...
        options.parse(argc,argv);
        ...
        options.parseFile("D:\Demo\config.ini");
        ...
        options.saveToFile("D:\Democonfig1.ini");

\end{verbatim}

in continuous block leads to all options gotten by argv and readed from the file
\verb|D:\Demo\config.ini| (either recognized and not recognized) will be saved in formatted
file \verb|D:\Demo\config1.ini|, and will be obtained again if programmer use ProgOptions::parseFile.

{\sf It is important:}

  Bouth ProgOptions::parseFile and ProgOptions::saveToFile methods return boolean value, which
indicate success or unsuccess of operatuion. In the case of ProgOptions::saveToFile, if
the method return false, than the system error has been occured and error information
is aviable via operation system errno interface.
Thus, the standard code fragment for storing options in file looks like the
next example:

\begin{verbatim}
 if (!optios.saveToFile(configFname)) {
    perror(options.argv(0));
    ....
 } 
 ....
\end{verbatim}

In the case of ProgOptions::parseFile the precidence rule analogous is not valid because
ProgOptions::parseFile return false in few different cases:
\begin{enumerate}
 \item system error occur,
 \item config file formated incorrectly,
 \item parsing of options is impossible.
\end{enumerate}
If the system error or bad format error occur, ProgOptions::parseFile
signals about situation by means of message which dispatchs to standard error stream.


\subsection{Format of ProgOptions config file}
\label{format}

   The file to parse must been formatted: 
\begin{enumerate}
\item The file must being a text formed by meaningful words,
      symbol-separators and comments. 
      The options to parse are meaningful words only.
\item The (single) meaningful word is <sequence>[<sequence>...] 
      where <sequence> is unbroken sequence of "visible" characters,
      sequence of visible characters, spaces and tabulators in double inverted commas,
      or sequence of sequences described. 

Example:
\begin{verbatim}

   aaa_bbb "aaa bbb" aaa" bbb ccc"

\end{verbatim}
will be interpreted as three separate words \verb|'aaa_bbb'|, \verb|'aaa bbb'| and \verb|'aaa bbb ccc'|

\item Comments are:
      \begin{enumerate}
      \item markers \verb|'/*'| and \verb|'*/'| and every characters standing between them;
      \item marker \verb|'//'| and part of string after it;
      \item the string begining from the \verb|'#'|.
      \end{enumerate}

Examples:

\begin{verbatim}
# This string will be ignored

  aaa /* This part of string will be ignored */ bbb

  ccc // This part of string will be ignored too
\end{verbatim}

\item The file must contain a marker being @"ProgOptions config file" or "@ProgOptions config file"
      (or analogous) as the first meaningful word in the first 10000 bytes of the file
      (hence, the first string of file could not be empty)

\item Effects:
   \begin{enumerate}
    \item if the commas is not closed, the bounds of word is end of line;
    \item if the double comma must be member of meanigful word, it may be used as '\verb|\"|'.
   \end{enumerate}

\end{enumerate}


\subsection {Trivial example of config file}

The trivial example which illustrate the structure of ProgOptions config file is follow:

%\label{example-of-config-file}

\begin{verbatim}

@"ProgOptions config file"

# The list of options: 

    --a          /* single option */
    --b c        // option with argument 
    --d="a b c"  // another option with argument

\end{verbatim}


\subsection { Usage of ProgOptions::parseString }

  The next additional possibility is that you may parse the string in C-style
with result the same as string has been readed from file.
The methodv ProgOptions::parseFile(const char* configFname, const char* executable="unspecified") do it. 
The small difference in acting of ProgOptions::parseFile and ProgOptions::parseString methods
is that the file marker \verb|'@ProgOptions Config File'| and symbol \verb|'#'| at the begining of string
must not have and have any special sence in latter case.


\section{ Programming environment conventions }
\label{environment}

Using Progoptions on Windows NT, you must:
\begin{enumerate}
  \item to define WIN32 macro before inclusion of ProgOptions.h header file;
  \item to use iostream, fstream, etc. standard headers instead iostream.h, fstream.h, etc. ones.
\end{enumerate}


\section{Changes}

\begin{itemize}
  \item[31.01.2002] - support of using \verb|--config <filename>| option described
  \item[24.01.2002] - warning concerning impossibility of \verb|--config=<filename>| syntax
                      for non-overloaded option \verb|--config| added.
  \item[03.01.2002] - next points updated in accordance with version 1.4.0 of ToolBox:
                      \begin{enumerate}
                      \item example in section \ref{base}
                            (include file \verb|GradSoft/ProgOptions| don't used from date);
                      \item descripton of configuration file parsing procedure in section \ref{format}
                            (the way to use '//','/*', and '*/' constructs in option body arised); 
                      \item environment conventions in section \ref{environment}
                            ("old" standard headers such as iostream.h, fstream.h and so on
                            must not be used from date).
                      \end{enumerate}
  \item[17.02.2001] - corrections, formal attributes of documentation set is added, susection abbout properties is added.
  \item[29.06.2000] - initial version. 
\end{itemize}



\end{document}
