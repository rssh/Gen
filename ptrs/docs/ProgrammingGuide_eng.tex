
\documentclass[10pt]{article}
%\usepackage{tex4ht}
\usepackage[OT1,T2A]{fontenc}
\usepackage[koi8-u]{inputenc}
\usepackage{graphicx}
\usepackage{verbatim}


\bibliographystyle{plain} 

\title{ Grad-Soft ptrs: Programmers Guide  
       \newline
       \small{DocumentId:GradSoft-PR-e-07.02.2002}
      }

% $Id: ProgrammingGuide_eng.tex,v 1.6 2002-03-19 05:51:24 rssh Exp $

\begin{document}

\maketitle{}

\tableofcontents

\section{ Introduction }

 Package {\bf ptrs-s } is devoted for advanced techniques of memory
 management in C++. 
 It includes set of smart pointers templates. Whith help of this
 templates developers can greatly reduce cost of attention to
  memory management issues
  during application development.

 For reading of this document base knoweldge of C++  and memory management
 is required; for introductory matherial you can look
 at 
 \cite{CPP-std}, \cite{Harvey}

\subsection{ About this package }

 This software is prodused by GradSoft company, Kiev, Ukraine.
Last version of this package is aviable on GradSoft Web site
\verb|http://www.gradsoft.com.ua/eng/|.

 You can free use this package and redistribute it with you programs,
according to license, which situated in file \verb|docs/LICENSE| inside
distributive of GradSoft C++ ToolBox.

 Commercial support of this package is aviable: call us for details.

\subsection{ About this document }

 This manual is writen for 1.5.0 version of GradSoft C++ ToolBox.
 Using ptr-s API from programmers point of view is described.
 Compilation and installation issues described in \cite{AD}

\section{ Class Hierarchy Description }

\subsection{ Common Place }

 If you familiar with concepts of \verb|smart pointers|, than you know, that
\verb|smart pointer| is a C++ class, which keep plain pointer, maybe
some infrastructure data and overload pointer dereferencing operators:
\verb|*, ->, ->*|.

 In addition  method \verb|get()| returning pointer itself is
provided for all types of smart pointer templates.

\subsection{ Exceptions }

 Package ptr-s introduce one exception into standart C++ exceptions
hierarchy. This is \verb|NullPointerExceptions|, which is raised when we
try to access to null pointer via \verb|operator->| or \verb|operator*| 
throught so-called {\bf safe} pointer classes.

Hierarchy:
\begin{verbatim}
 std::runtime_exception
   |
   *--->NullPointerException
\end{verbatim}

\subsection{ safe\_ptr }

 \verb|safe_ptr| is just wrapper arround pointer, which during applying
of accessors throw exception \verb|NullPointerException| instead switching
program to undefined behaviour.
 Typical pattern of usage is keep in \verb|safe_ptr| pointers to data,
owned by other subsystems. 

Example:
\begin{verbatim}
void myFun(Something* x) throw (std::exception)
{
 safe_ptr<Something> sx(x);
 sx->do();
}
\end{verbatim}
can be used instead:
\begin{verbatim}
void myFun(Something* x) throw (std::exception)
{
 if (x==NULL) throw std::runtime_exception(string("x is null"));
 x->do();
}
\end{verbatim}

\subsection{ safe\_auto\_ptr }

 This is full analog of \verb|std::auto_ptr| with one change: 
throw \verb|NullPointerException| during dereferencing of NULL pointer.

Note: we does not include ANSII C++ template methods for \verb|std::auto_ptr|,
since luck of compiler support for this language feature. 

\subsection{ owned\_ptr }

As you can guess from name of template, \verb|owned_ptr| is a pointer which
'owned' by some entity. What this mean: in \verb|owned_ptr| we hold pointer
itself and boolean flag which indicate 'ownity'. If ownity set to true,
than destructor of holded class is called during \verb|owned_ptr| destruction.

Safety of \verb|owned_ptr| are denoted by second template parameter, which
must be one of trait structures \verb|ptr::safe| or \verb|ptr::unsafe|.
\begin{itemize}
 \item \verb|owned_ptr<T,ptr::safe>| - for safe owned pointer.
 \item \verb|owned_ptr<T,ptr::unsafe>| - for unsafe owned pointer.
\end{itemize}

Unlike other pointer templates, owned pointer have no overloaded
\verb|operator=| , instead we have method set with to parameters: one is
pointer to set, other - boolean flag, which indicate passing of ownership.

Examples:
\begin{verbatim}
 MyClass* px = new MyClass();
 owned_ptr<MyClass,ptr::safe> a(px,true);
 MyClass* py = new MyClass();
 owned_ptr<MyClass,ptr::safe> b(py,true);
 a.set(b,true); 
\end{verbatim}
 \verb|py| now owned by \verb|a|, \verb|b| point to \verb|py| but can be 
destructed without affecting of \verb|py|. \verb|px| is destructed.

\begin{verbatim}
 owned_ptr<MyClass,ptr::safe> a;
 MyClass* px = new MyClass();
 owned_ptr<MyClass,ptr::safe> b(px, true);
 a.set(b,false); 
\end{verbatim}
 \verb|a| now point to \verb|px|, but \verb|px| is still owned by \verb|b|. 

\begin{verbatim}
 owned_ptr<MyClass,ptr::safe> a;
 MyClass* px = new MyClass();
 owned_ptr<MyClass,ptr::safe> b(px, false);
 a.set(b,false); 
\end{verbatim}
 \verb|a| and \verb|b| now points to \verb|px|, but \verb|px| is not owned
by anything. 

\begin{verbatim}
 MyClass* px = new MyClass();
 owned_ptr<MyClass,ptr::safe> a(px,false);
 MyClass* py = new MyClass();
 a.set(py,false); 
\end{verbatim}
 \verb|px| is leaked.

\subsection{ counted\_ptr }

\subsubsection{ Theory }

 \verb|counted_ptr| incapsulate well-known idiom of reference counting: 
object is living while it's pointed, so:
\begin{itemize}
 \item let's keep pair of object and counter value. 
 \item when we create pointer to object let's increment counter.
 \item when we desctruct pointer to object let's decrement counter.
 \item when counter became zero, i. e. last pointer, which point to our object
 was destructed, let's destruct object.
\end{itemize}

 This idiom greatly simplicified the task of memory management for acyclic 
structures.

\subsubsection{ Usage }

 Our \verb|count_ptr| template have 2 template parameters: first is type of
wrapped object, second - safety class, one of \verb|ptr::safe| or 
\verb|ptr::unsafe|.

 In addition to usial pointer operations, counted-ptr provide additonal
method: \verb|counter_ptr<T,safety>::assign(T* new)|, which change value
of internal shared pointer.

examlple:
\begin{verbatim}
counted_ptr<MyClass,ptr::safe> a(pA);
counted_ptr<MyClass,ptr::safe> b=a;
b.assign(pB).
\end{verbatim}
 now \verb|*pA| is destroyed and both \verb|a| and |b| point to \verb|pB|.
  

\subsubsection{ Limitation }

\begin{itemize}
  \item our \verb|count_ptr| template is not thread-safe. 
 We provide thread-safe counted pointer as \verb|count_mt_ptr| in 
 Threading package.
  \item reference counting work only for acyclic structures, for cyclic
 memory objects you must use some garbage collection technique.
\end{itemize}


\section{ Changes }

\begin{itemize}
 \item 19.03.2002 - first english version complete.
 \item 07.03.2002 - created.
\end{itemize}

 
\bibliography{Bib} 

\end{document}
