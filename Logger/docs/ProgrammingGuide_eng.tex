
\documentclass[10pt]{article}
%\usepackage{tex4ht}
\usepackage[OT1,T2A]{fontenc}
\usepackage{graphicx}
\usepackage{verbatim}

%\Contribute{TITLE}{\def\LaTeX{LaTeX}}
\title{ Logger: Programming Guide  
       \newline
       \small{DocumentId:GradSOft-PR-09.08.2000-v1.2.0 }
      }

% $Id: ProgrammingGuide_eng.tex,v 1.12 2002-06-10 18:08:26 rin Exp $

\begin{document}

\maketitle{}

%\tableofcontents

\section{ Introduction }

 Logger is a C++ component, which allow easy add loggin abilities
  to you application 
  and organize event-depended call of user functions. 

 This document is an unformal description of package, for full specification,
 please, see API reference. 
 (\verb|www.gradsoft.kiev.ua/common/ToolBox/Logger/API/index.html|).

\section{ General description }

 Using of Logger must be  follow next pattern:

 \begin{itemize}
   \item Application programmer must create object of type Logger and set it
 configurable parameters.
   \item Object of type Logger afford to application programmer virtual streams for messages passing. Five types of messages are predefined: 
    Debug, Info, Warning, Errors, Fatals.
   \item For writing to this streams programmer uses next logger attributes:
    \begin{itemize}
        \item  Logger::debugs(),  
        \item  Logger::infos(),  
        \item  Logger::warnings(),  
        \item  Logger::errors(),  
        \item  Logger::fatals(),  
    \end{itemize}
    by the next way:
    \begin{verbatim}
      logger.errors() << "This is error:" << 334 << endl;
    \end{verbatim}
    After execution of this code string \verb|"This is error:334"| will be 
  writed to log file, and if callback function for errore was set, it will
  be called with argument \verb|"This is error:334"|
   \item 
     For setting callback function, which called when message of some type
    is outted, application programmer must use method:
     Logger::setCallback
 \end{itemize}

\subsection{Compile time settings}

  
   It is possible to enable or disable output to logger streams in
  compile time by setting next preprocessor symbols to values true of false:
  \begin{itemize}
    \item \verb|LOG_DEBUG_ENABLE| - output to debug stream (i. e. to
   \verb|logger.debugs()| is enabled. When \verb|LOG_DEBUG_ENABLE| set to
   true, expression \verb|logger.debugs() << msg << endl()| is evaluated as
   was described in previous section. Otherwise, this statement reduct to
   "nothing-do" statement, which must be eliminated by smart C++ compiler.
   By default \verb|LOG_DEBUG_ENABLE| is set to \verb|false|.
    \item \verb|LOG_INFO_ENABLE| - enable output to infos stream (i. e. logger.infos() ). It is set to \verb|true| by default.
    \item \verb|LOG_WARNING_ENABLE| - enable output to warnings stream. 
 Default value is true.
    \item \verb|LOG_ERROR_ENABLE| - enable output to error stream. 
 Default value is true.
    \item \verb|LOG_FATAL_ENABLE| - enable output to fatals stream. 
 Default value is true.
 \end{itemize}

\subsection{Run time settings}

 Also exists next  run-time Logger settings:
\begin{itemize}
  \item - file name for logger output.  Appropriative method is:
\begin{verbatim}
void Logger::setOutputFile(const char* fname) throw Logger::IOException
\end{verbatim}
  This method generate exception \verb|IOException| on unsuccess.
  \verb|IOException::what()| contains error message.
  \item - are we want additonally output all messuges to user terminal ?
\begin{verbatim}
void Logger::setDuppedToStderr(bool x) 
\end{verbatim}
  Default value is false. In addition you can set this option as parameter of
Logger constructor.
  \item - are we want generate syslog messages to store messages in system journal ?
\begin{verbatim}
void Logger::setSyslogOutput(bool x) 
\end{verbatim}
   Default value is true. Note, that under Windows NT this option have no effect.
\end{itemize}

  
\section{ Example }
  which illustrate Logger use is follow:


\begin{verbatim}

#define LOG_DEBUG_ENABLE true

#include <GradSoft/Logger.h>

void debug_callback(const char* msg)
{
 cerr << "debug_callback:" << msg << endl;
}

int main(int argc, char** argv)
{
 try {
  GradSoft::Logger logger("file.log");
  logget.setCallback(GradSoft::Debug,debug_callback);
 
  logger.debugs() << "debug output 1 for " << argv[0] << endl; 
 }catch(Logger::IOException){
   cerr << "can't open log file" << endl;
   return 1;
 }

 return 0;
}

\end{verbatim}

\section{ Using Logger in multithreaded applications }
 
 You can use Logger in muiltithreaded applications: all Logger methods
are thread-safe. But during using of logger output streams via 
\verb|operator<<| exists potential problem of interference of messages 
from different streams. For preventing this we reserve mutex for each logger
stream and define class - lock guard of this mutex which is lock mutex
on creation and unlock on destruction.

 So, we reccomend use next code fragment as codding pattern:
\begin{verbatim}
 {
  Logger::DebugLocker guard(logger.debugs());
  Logger.debugs() << "print " << "what " << "you " << "want" << endl;
 }
\end{verbatim}

 Now more formal definition and naming scheme for locking classes:
 
 For each event type \verb|Xxx| class \verb|Logger::XxxLocker| is defined.
The methods of \verb|Logger::XxxLocker| are:
\begin{itemize} 
  \item \verb|XxxLocker(XxxStream&)| - own mutex which control output to 
  \verb|xxxs()|.
  \item \verb|~XxxLocker()| - free this mutex. 
\end{itemize} 
 In case, when appropriative debug stream is disabled, lock class is reduces
to empty class with empty operations.


\section{ Programming Environment Conventions }

\begin{enumerate}
  \item You standart C++ library must support \verb|string| type.
  \item Few autoconf-derived macroporocessor variables are defined in file 
  \verb|LoggerConfig.h| (or \verb|LoggerConfingNT.h| for Windows)
 which is generated during Logger installation. 
  {\em before} inclusion of file Logger or Logger.h
  Potentially names of this macroses can potentially conflict with autoconf names  of other packages or you main program.
  To prevent this, we reccomend use \verb|#ifdef quards| for you autoconf macroses:
\begin{verbatim}
#ifdef HAVE_Xxx  
#undef HAVE_Xxx  
#endif
\end{verbatim}
  \item Using Logger on Windows NT, you must:
    \begin{enumerate}
      \item to define WIN32 macro before inclusion of Logger.h header file;
      \item to use iostream, fstream, etc. standard headers instead iostream.h, fstream.h, etc. ones.
    \end{enumerate}
\end{enumerate}

\section{ Changes }

\begin{itemize}
 \item [03.01.2002] - updated in accordance with GradC++ToolBox 1.4.0
 \item [03.07.2001] - changed example: removed using of deprecated header 
   \verb|GradSoft/Logger|
 \item [02.06.2001] - changed programming environment and added 
 sections about 1.2.0 features.
 \item [18.02.2001] - review, added formal document attributes.
 \item [09.08.2000] - initial revision.
\end{itemize}


\end{document}
