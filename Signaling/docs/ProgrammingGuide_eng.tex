\documentclass[12pt]{article}
%\usepackage{tex4ht}
\usepackage[OT1,T2A]{fontenc}
\usepackage{graphicx}
\usepackage{verbatim}

\title{Signaling : Programmer's guide.
       \newline
       \small{DocumentId:GradSoft-PR-27.09.2001-v1.0.2}
      }
% $Id: 

\begin{document}

\maketitle{}

\section{ Introduction }
   Component \verb|Signaling| is intended for POSIX signal handling. In POSIX 
system signals are used for processing non-ordinary events which are required
defined handling.
  Events could be called by process, user or kernel. In some sense POSIX 
signal is program implementation of hardware interruption. 
The set of all signals used in system is enumerate in header-file <signal.h>.
In <signal.h> every signal has its unique name and number. For getting more
detail information about signal you can treat to corresponding sources.
( for example for UNIX see {\bf man signal}) 

\section{ General description of functionality mechanism }
     Before using in program component Signaling  it has  
necessary been written signal's handler. Signal handler is represented by defined
class which inherits from abstract class {\bf SignalHandler}. 
Signal implementation must be overloaded in method {\bf void handler(
int insignal\_number) } of {\bf SignalHandler} class. Only one parameter is
sent to this method which is the signal internal number for which handler is
called. Signal handler is set by wrapped class Signaling. Signature of
Signaling constructor is \\
{\bf Signaling( unsigned long  sigset, SignalHandler\& handler )}\\
where \\
\verb|sigset| - the set of signal handlers.(see below)\\
\verb|handler| - class derived from SignalHandler.\\
For technical problem component Signaling has its own internal name and number
for signal which is described in class SignalSet. Internal name coincides with
standard name plus add prefix "{s}".\\
{\it
  {\bf Attention: } For component Signaling use only constant
   which is defined in class SignalSet.
}\\
For defining the set of signals use statement | {"OR"}. If you need to define
all the signals then you can use internal constant  
\verb|SignalSet::sENABLE_ALL|.
(Constant  \verb|SignalSet::sENABLE_ALL| is excluded  signals which 
could not be caught). Amount all signals is defined in constant 
\verb|SignalSet::sNSIG|.

\section{ Using }
At first define signal handler:
\begin{verbatim}
/* File demo_handler.h */
. . .
class demo_handler : public SignalHandler {
public :
      virtual void handler(int signal) {
        cerr << "Catch signal :" << signal << endl;
      }
}
. . .
\end{verbatim}
Set the handler "demo\_handler" for handle all signals.
\begin{verbatim}
#include <demo_handler.h>
. . .
{
    Signaling(SignalSet::sENABLE_ALL, demo_handler);
    . . .
}
. . .
\end{verbatim}
For more detail information see directory \verb|demo| in distribution.
\section{Requirements.}
\begin{itemize}
  \item Before using Signaling it has been needed to define macro 
     {\bf HAVE\_SIGACTION} or {\bf HAVE\_SIGNAL} depending on which function
     will be used for signal handling {\bf sigaction}  or {\bf signal}.
  \item If compiler supports \verb|namespaces| then macro 
     {\bf HAVE\_NAMESPACES} will have been defined.
\end{itemize}
\end{document}
