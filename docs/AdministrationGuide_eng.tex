\documentclass[10pt]{article}
\usepackage[OT1,T2A]{fontenc}
\usepackage{graphicx}

\title{ GradC++ ToolBox: Administration Guide  
       \newline
       \small{DocumentId:GradSoft-AD-e-04.09.2000-v1.2.0}
      }

% $Id: AdministrationGuide_eng.tex,v 1.32 2003-05-08 10:31:30 rssh Exp $

\newcommand{\version}{1.6.3}

\begin{document}

\maketitle{}

\tableofcontents

\section{Introduction}

\subsection{Subject}

GradC++Toolbox is a joined package composed by several platform-independet components
designed to resolve standard widely-spread problem such as:
\begin{list}{-}{}
  \item implementation of smart pointer templates;
  \item parsing of the command line;
  \item writing debug or info messages;
  \item reading file directories;
  \item starting and stopping of threads and so on.
\end{list}
We beg to propose this package for programm builders,
which want to save them from some standard problems
or need to produce portable sourse code for UNIX and for Windows NT.
This file describe a routine of obtaining, installing and using of version \version of package.
This version GradC++ToolBox 1.6.3 consists from following components:
\begin{enumerate}
  \item ptrs - set of smart pointer templates;
  \item ProgOptions - class for high-level handling of program options;
  \item Threading - set of classes to design of multithreading programs;
  \item Logger - class for logging support;
  \item DirectoryContainer - set of classes to read file directories;
  \item DynamicModules - component for shared libraries dynamic loading.
  \item ServiceOptions - specialization of ProgOptions for CORBA services;
\end{enumerate}
You may install by choice:
\begin{enumerate}
  \item a whole set of package components;
  \item a set of all non-CORBA components which are (today) ptrs, ProgOptions, Logger, Threading, DirectoryContainer and DynamicModules.
\end{enumerate}
Besides, by handle dependencies between some components, some single components can be installed too.

\subsection{Source}

Current version of GradC++ToolBox is always avaible on builder's website: http://www.gradsoft.com.ua/eng \\

\noindent
No payment if license is satisfied (see file {\bf license} from delivery set).


\subsection{Development team}            

GradC++ToolBox is designed and distributed by GradSoft company, GradSoft home page is: http://www.gradsoft.com.ua/eng


\section{ Installation }

\subsection{ General information }

GradC++ToolBox is distributed in the form of source code,
its installation process consists from following steps:
\begin{enumerate}
\item compilation of source code in user environment;
\item coping of include files and libraries generated into directories defined by user.
\end{enumerate}
You may:
\begin{enumerate}
\item compile code;
\item copy ("install") include files and libraries into directories defined by user;
\item remove installed files ;
\item remove automatically generated files (such as *.obj and *.lib)
\end{enumerate}
using command {\bf make} with options correspondent (see below for details).


\subsection{ Software needed }

\begin{enumerate}
 \item CORBA ORB (not needed, if you not use ServiceOptions): 
  \begin{itemize} 
    \item Unix: one of omniORB-3.0 or higher, TAO-1.1 or higher, ORBacus 4.0.1 or higher, MICO-2.3.9 or highter.
    \item WindowsNT: one of omniORB-3.0 or higher, TAO-1.2, ORBacus 4.0.1 or higher 
  \end{itemize}
 \item C++ compiler: 
  \begin{itemize} 
   \item Unix:  gcc-2.95.2 or higher, SunProc C++ 6.2
   \item Windows NT: Microsoft Visual C v. 6.0 or higher.
  \end{itemize}
 \item make:
  \begin{itemize} 
   \item Unix:  gnu make is necessary
   \item WindowsNT: nmake from  MSVC++
  \end{itemize}
\end{enumerate}

\subsection{ Installing for UNIX}


\begin{enumerate}
 \item Make sure that necessary software are installed and are in working state.
 \item Extract files from archive Gen-1.6.3.tar.gz or Gen-1.6.3.zip in some catalog. Let's name
 this catalog \verb|<project_root>|
 \item cd to \verb|<project_root>|.
 \item Run configure with command \verb|./configure [options]| 
 (options list is accessible via command \verb|./configure --help|; 
 for example, you can set \verb|--prefix=<smth>| option to set directory of installation 
 or \verb|--with-corba=<yes/no>| option to set volume of compilation and installation).
 \item Start compilation process with command \verb|gmake|
 \item Become superuser. 
 \item Run install process with command \verb|gmake install| 
 \item For deinstallation you can use command \verb|gmake uninstall| 
\end{enumerate}


\subsection{ Installing for Windows NT }

\begin{enumerate}
 \item Check, that all necessary software is installed and are in working state.
       Requirements to use MSVC are
       \begin{enumerate}
       \item Environment variables INCLUDE and LIB must be defined and must contain pathes
             to MSVC include files and MSVC libraries correspondingly.
       \item location of {\bf nmake}, {\bf cl} and {\bf link} utilities must be reflected
             in the environment variable PATH.
       \end{enumerate}
 \item Extract files from archive Gen-\version.tar.gz or Gen-\version.zip in some directory
       (let's name of  this catalogue \verb|<project_root>|).

 \item Edit file {\bf environment.nt.mak} in directory \verb|<project_root>\config\Win32|
       and, in the case you want to use ServiceOptions,
       analogous file {\bf environment.nt.mak} in directory
       \verb|<project_root>\CORBA\config\Win32|;
       set values of next nmake variables:

       \paragraph{I.} \verb|<project_root>\config\Win32\environment.nt.mak|:

       \begin{itemize}

       \item INSTALL\_DIR (installation root) - include files and libraries supplied
             will be quartered in subdirectories {\bf install/GradSoft} and {\bf lib}
             of this directory.

       \end{itemize}

       \paragraph{II.} \verb|<project_root>\CORBA\config\Win32\environment.nt.mak|:

       \begin{itemize}

       \item ORB (the "name" of ORB) - it must be ORBacus, or OmniORB, or TAO
             in concordance with ORB used

       \item ACE\_ROOT - ACE root (in the case of using TAO as the ORB)
       \item TAO\_ROOT - TAO root (in the case of TAO is placed outside of ACE directory tree)
       \item OMNI\_ROOT - omniORB root (in the case of using omniORB)
       \item OOC\_ROOT - ORBacus root (in the case of using ORBacus)
       \item MICO\_ROOT - MICO  root (in the case of using MICO)
       \item PTREADS\_ROOT - location of pthreads-win32 source distributive (in the case of using MICO)
       \end{itemize}

 \item To compile package go to \verb|<project_root>| and type {\bf make build}
       if you want to compile no CORBA components,
       or {\bf make with-corba build} if you want to compile ServiceOptions too.
 
 \item To install include files and libraries go to \verb|<project_root>|
       and type {\bf make install}
       if you want to install no CORBA components,
       or {\bf make with-corba install} if you want to install ServiceOptions too.
   
 \item To uninstall installed files go to directory \verb|<project_root>| and type 
        {\bf make uninstall} or {\bf make with-corba uninstall} respectively.
 
 \item To remove automatically generated files, such as *.obj, *.exe and *.lib,
       from \verb|<project_root>| and its subdirs go to directory \verb|<project_root>| and type
       {\bf make clean} or {\bf make with-corba clean} respectively.
   
 \item For compilation, installation, deinstallation or cleaning of single (apart) subpackage move over
       to directory suitable (one of next:
       \begin{list}{}{}
       \item \verb| <project_root>\ProgOptions| 
       \item \verb| <project_root>\CORBA\ServiceOptions| 
       \item \verb| <project_root>\ptrs| 
       \item \verb| <project_root>\Threading| 
       \item \verb| <project_root>\Logger|
       \item \verb| <project_root>\DynamicModules|
       \item \verb| <project_root>\DirectoryContainer| )
       \end{list}
       and use command {\bf make} with options {\bf build}, {\bf install}, {\bf uninstall} and {\bf clean} correspondingly

 \end{enumerate}

\paragraph{Remark:}
Using {make install} without compilation of the package 
leads to compilation of libraries will be started automatically
but test examples will not be compiled.

\section{ Testing }

\subsection{ Common information }

Demo examples, included in GradC++ToolBox,
are related to using examples in Programmers Guides for each package commponent.
It is necesorry to compile project before starting demos.

\subsection{ Procedure }

\subsubsection{ ProgOptions }

\begin{enumerate}
 \item Read ProgrammingGuide in file \newline
 \verb|<project_root>|/ProgOption/docs/ProgrammingGuide\_eng.pdf
 \item cd to directory \verb|<project_root>|/ProgOptions/demo
 \item run  ProgOptionsDemo few times with next options:
 \begin{verbatim}
  --help
  --qqq
  --zzz
  --zzz <something>
  --zz1 <something> 
  --zz3
  <something>
  --config config.ini
 \end{verbatim}
 \item check, that program output is expected
\end{enumerate}

\subsubsection{ ServiceOptions }

\begin{enumerate}
 \item Read ProgrammingGuide in file \newline
 \verb|<project_root>|/CORBA/ServiceOptions/docs/ProgrammingGuide\_eng.pdf
 \item Go to directory \verb|<project_root>|/CORBA/ServiceOptions/demo
 \item For UNIX: make tests, thereto:
       \begin{enumerate}
       \item call {\bf ./configure} with options appropriate
       \item call {\bf gmake} (or {\bf gmake build})
       \end{enumerate}
 \item Start HelloWorldServer with option \verb|--help|; read help message.
 \item Start HelloWorldServer with option \verb|--ior-stdout|;
 make sure that Interoperable Object Reference is displayed on standard output 
 \item Start HelloWorldServer with options: \verb|--ior-file-HelloService hello.ref|
 Make sure that Interoperable Object Reference is placed into file hello.ref 
 \item Start HelloWorldServer with options: 
  \begin{itemize}
    \item \verb|-OAport 1100| if you use ORBacus
    \item \verb|-ORBport 1100| if you use TAO-1.1
    \item \verb|-ORBEndpoint iiop://:1100| if you use TAO-1.2
    \item \verb|-ORBpoa_iiop_port 1100| if you use omniORB
    \item \verb|-ORBIIOPAddr=inte:<hostname>:1100| if you use MICO
  \end{itemize}
  (note port 1100 must be not busy by some other application; of course, you can
  use any port number instead 1100)
 \item Start HelloWorldClient with options: 
  \begin{itemize}
    \item if you use omniORB3 or TAO 1.2 or ORBacus 4.x : 
   \newline\verb|-ORBInitRef HelloService=corbaloc::127.0.0.1:1100/HelloService|
    \item if you use TAO-1.1: 
   \newline\verb|-ORBInitRef HelloService=iiop://127.0.0.1:1100/HelloService|
  \end{itemize}
 \item Make sure that string "Hello World" is printed on you terminal
 \item Check you NameService daemon
 \item Start program HelloWorldServer with option \verb|--with-naming|
 \item Start HelloWorldClient with option \verb|--with-naming|
 \item Make sure that string "Hello World" is printed on you terminal
\end{enumerate}

\subsubsection{ Logger }

\begin{enumerate}
 \item Read ProgrammingGuide in file \newline \verb|<project_root>/Logger/docs/ProgrammingGuide_eng.pdf|
 \item Move out to directory \verb|<project_root>/Logger/demo|
 \item Run demo executable
 \item Make sure that file.log is generated and its content correspond with usage rules
 described in programming Guide.
\end{enumerate}

\subsubsection{ DirectoryContainer }

\begin{enumerate}
 \item Read ProgrammingGuide in file \newline \verb|<project_root>/DirectoryContainer/docs/ProgrammingGuide_eng.pdf|
 \item Move out to directory \verb|<project_root>/DirectoryContainer/demo|
 \item Run demo executable
 \item Make sure that file.log is generated and its content correspond with usage rules
 described in programming Guide.
\end{enumerate}

\subsubsection{ ptrs }
 \begin{enumerate}
 \item Read ProgrammingGuide in file \newline \verb|<project_root>/ptrs/docs/ProgrammingGuide_eng.pdf|
  Move out in series to:
  \begin{enumerate} 
  \item \verb|project_root/ptrs/demo/safe_ptr|
  \item \verb|project_root/ptrs/demo/owned_ptr|
  \end{enumerate}
  \item Make sure that executable files is present and is in working state.
  \end{enumerate}

\subsubsection{ Threading }
  Move out in series to:
  \begin{enumerate} 
  \item \verb|project_root/Threading/demo/Container|
  \item \verb|project_root/Threading/demo/DeleteThis|
  \item \verb|project_root/Threading/demo/Hairdresser|
  \item \verb|project_root/Threading/demo/Philosophs| 
  \item \verb|project_root/Threading/demo/Stres|
  \item \verb|project_root/Threading/demo/ThreadServices|
  \end{enumerate}
  Make sure that executable files is present and is in working state.

\subsubsection{ DirectoryContainer }

\begin{enumerate}
 \item Read ProgrammingGuide in file \newline \verb|<project_root>/DynamicModules/docs/ProgrammingGuide_eng.pdf|
 \item Move out to directory \verb|<project_root>/DynamicModules/demo|
 \item Run Hello executable
 \item Make sure that output is:
\begin{verbatim}
Hi! I'm Hello1
Hi! I'm Hello2, name was xxx
\end{verbatim}
\end{enumerate}



\section{ Using }

\paragraph{A.}
Windows NT:
WIN32 macros must be defined befor including of any include file from GradC++ToolBox package

\paragraph{B.}
Programs, which use some of GradC++ToolBox components must be compiled with the next libraries:
\newline
\newline
For Unix:
\newline

\begin{tabular}{|l|l|}\hline

  {\em component}
& {\em staic library} \\ \cline{2-2}
& {\em dinamic library} \\ \hline

  ProgOptions
& libProgOptions.a \\ \cline{2-2}
& libProgOptions.so \\ \hline

  ServiceOptions
& libServiceOptions.a, libProgOptions.a \\ \cline{2-2}
& libServiceOptions.so, libProgOptions.so \\ \hline    

  Threading
& libThreading.a \\ \cline{2-2}
& libThreading.so \\ \hline

  Logger
& libLogger.a, libThreading.a  \\ \cline{2-2}
& libLogger.so, libThreading.so \\ \hline

  DirectoryContainer
& libDirectoryContainer.a, libThreading.a  \\ \cline{2-2}
& libDirectoryContainer.so, libThreading.so \\ \hline

  DynamicModules
& libDynamicModules.a  \\ \cline{2-2}
& libDynamicModules.so \\ \hline

\end{tabular}
\newline
\newline
\newline
 For Windows NT:
\newline

\begin{tabular}{|l|l|}\hline

  {\em component}
& {\em staic library}\\ \hline

  ProgOptions
& ProgOptions.lib \\ \hline

  ServiceOptions
& ServiceOptions.lib, ProgOptions.lib \\ \hline    

  Threading
& Threading.lib \\ \hline

  Logger
& Logger.lib, Threading.lib \\ \hline

  DirectoryContainer
& DirectoryContainer.lib \\ \hline

  DynamicModules
& DynamicModules.lib \\ \hline

\end{tabular}
\newline
\newline


\section{ Dependencies }

Take into account if want to treat single component of package:
\begin{itemize}
\item Compile ProgOptions before ServiceOptions
\item Compile Threading before Logger and DirectoryContainer
\item configure ptrs before compiling threading.
\end{itemize}

I. e. dependency diagram is shown in next picture:
\begin{verbatim}
 ptrs                  ProgOptions
  |                        |
  *-- Threading            *--ServiceOptions
         |
         |
         *--Logger
         |
         *--DirectoryContainer
\end{verbatim}

\section{ Changes }

\begin{itemize}
  \item[07.05.2003] 
                   \begin{itemize}
                     \item release number changed (1.6.0 -> 1.6.3)
                     \item added information about MICO.
                   \end{itemize}
  \item[29.08.2002] 
                   \begin{itemize}
                     \item release number changed (1.5.0 -> 1.6.0)
                     \item version of supported Sun CC compiler changed.
                   \end{itemize}
  \item[28.04.2002] 
                   \begin{itemize}
                     \item release number changed (1.4.2 -> 1.5.0)
                     \item reflected adding of \verb|ptrs| component.
                     \item added dependency diagram.
                   \end{itemize}
  \item[06.02.2002] - release number changed (1.4.1 -> 1.4.2)
  \item[24.01.2002] - release number changed (1.4.0 -> 1.4.1)
  \item[27.12.2001] - overpatching reflecting next stuffs: 
                      \begin{enumerate}
                      \item alteration of installation process for Windows NT;
                      \item adding of omniORB and TAO support for Windows NT;
                      \item adding new component DynamicModules.
                      \end{enumerate}
 \item[18.09.2001] - added info about DirectoryContainer component
 \item[04.07.2001] - changed port number in CORBA demo explanation.
 \item[29.04.2001] - fixed few minor typos.
 \item[14.02.2001] - added detailed information about ORB configuration options
 \item[04.09.2000] - first revision
\end{itemize}

\end{document}
