\documentclass[10pt]{article}
%\usepackage{tex4ht}
\usepackage[OT1,T2A]{fontenc}
\usepackage[koi8-u]{inputenc}
\usepackage{graphicx}
\usepackage{verbatim}

%\Contribute{TITLE}{\def\LaTeX{LaTeX}}

\title{ SMTPmailer: tool for sending messages }

\begin{document}

\maketitle{}

\tableofcontents

\section{Introduction}
SMTPmailer is a class that contains the number of methods which  are useful for
sending messages to reciever or group of recivers. For sending mails SMTPmailer
uses a uniform interface for different platforms(U*NIX/Windows). Using of 
the uniform helps creating well-porting program for heterogeneous operation
system.

\section{Small example}

\hrule \vspace {10pt}

\begin{verbatim}

#include <GradSoft/SMTPmailer.h>

#include <iostream>
#include <string>

using namespace std;

int main(int argc, char* argv[])
{
   const int addr_len = 50;
   char addr[addr_len];
   if (argc < 3) {
      cerr << "Usage : SMTPmailerDemo 
	      <snmp-server ip address> <e-mail address>" << endl;
      return 0;
   }
   if (strlen(argv[2])<addr_len)  strncpy(addr,argv[2],addr_len);
   else return 0;
   SMTPmailer mailer(argv[1]);
   try {
      mailer.beginSend("AutomaticSMTPmailer@gradsoft.com.ua",
					(const char*) addr );
      mailer << "Hello world !!!\nBye !!!" << "\n....\n";
      mailer << "..\n.\n.";
      mailer.endSend();
   } catch(const SMTPmailer::SMTPError& ex) {
      cerr << "Error:" <<  ex.error << endl;
   }
   return 0;
}

\end{verbatim}

\hrule \vspace {10pt}
\section{Changes}
\end{document}
