
\documentclass[10pt]{article}
%\usepackage{tex4ht}
\usepackage[OT1,T2A]{fontenc}
\usepackage[koi8-u]{inputenc}
\usepackage{graphicx}
\usepackage{verbatim}

%\Contribute{TITLE}{\def\LaTeX{LaTeX}}
\title{ DynamicModules: Programmer Guide
        \newline
        \small{DocumentId:GradSoft-PR-r-01.11.2001-v1.0.0}
}
\author{ Ruslan Shevchenko }

% $Id: ProgrammingGuide_eng.tex,v 1.5 2003-09-23 09:42:22 qwerty Exp $

\begin{document}

\maketitle{}

\tableofcontents

\section{ Introduction }

 {\bf DynamicModules}  is cross-platform component written in C++,
 designed for shared libraries dynamic loading 
 (\verb|.so| for UNIX or \verb|.dll| for Windows )

 Package {\bf DynamicModules} is designed and supported by GradSoft company
 and supplied in source code.
 Developer home webpage is http://www.gradsoft.kiev.ua/

% This document is informal package description. Full specification you can
% find in API description (www.gradsoft.kiev.ua/common/Products/Toolbox/DirectoryContainer/API).

\section{ Common information }

   DynamicModules package allows to organize program building as set consisting of  
head module and shared libraries loading while execution.  
 
 Loadable module developer just supplies the module and set of header
 files.

 There is must be static object in the module body, inherited  from 
 \verb|DynamicModule| class or \ verb|DynamicModuleTag| class. 

 Program, which is using shared libraries should be compiled with
 \verb|[lib]DynamicModules.{a,so,lib}| library and should use \verb|DynamicModules|
 class  methods (load, unload, getModule) described in the library.

\section{ Trivial example }

 \subsection{Components}

 Trivial example described here consist of followins components:
 \begin{enumerate}
 \item common header file HelloInterface.h,
       including DynamicModules.h header file in one's turn
 \item loadable module HelloModule.cpp
 \item head programm Main.cpp
 \end{enumerate}

 \subsection{Common Header File}

 This file contains data used by both library and client:
 \begin{enumerate}
 \item class HelloInterface - this is interface through which head programm
       will operate for its specific purpose.
 \item class HelloModule - this is interface through which head programm
       will call to root object of the service 
 \end{enumerate}

\begin{verbatim}
#include <GradSoft/DynamicModules.h>

/**
 * Interface for direct work of programm:
 **/
 class HelloInterface
 {
 public:
   virtual ~HelloInterface() {}
   virtual void hello() = 0;
 };

/**
 * Interface for call to root object of the service:
 **/
 class HelloModule: public GradSoft::DynamicModule
 {
 public:
  /**
   * return reference to HelloInterface instance:
   **/
   virtual HelloInterface* create(const char* name) = 0;
 };
\end{verbatim}

\subsection{Loadable module}

This file contains realisation for HelloModule and HelloInterface:
  
\begin{verbatim}

#include <HelloInterface.h>  
#include <iostream>

// realize direct functionality:

class Hello: public HelloInterface
{
public:

  Hello(const char* x) {}
  virtual ~Hello() {}
  virtual void hello() 
  {
    std::cerr << "Hi! I'm Hello" << std::endl;
  }
};

// realize root object (or the Factory):

class Hello1Module: public HelloModule
{
public:

////  
// overload next methods of DynamicModule parent class:
////
   
  // service name
  const char* name() const { return "Hello"; }

  // version information
  int versionMajor() const { return 1; }
  int versionMinor() const { return 0; }
  int versionSubMinor() const { return 0; }

  // author
  const char* author() const { return "Grad-Soft LTD"; }

////
// realize specific method of HelloModule:
////

  HelloInterface* create(const char* args)
  {
    return new Hello(args);
  }

};

// create module:
Hello1Module tagHelloModule;

// export object:
EXPORT_OBJECT(tagHelloModule)

\end{verbatim}

\subsection{Head programm}

\begin{verbatim}
#include <HelloInterface.h>
#include <memory>
#include <iostream>

using namespace GradSoft;

int main()
{
 try {

  // loading Hello module from current directory
  DynamicModule& dmHello = DynamicModules::load("tagHelloModule","./Hello");

  // casting to type HelloModule in order to call HelloModule::create

  // We need firstly to check type manually, because static_cast doesn't provide the check 
  // Certainly, to use dynamic_cast is more elegant way, but dynamic_cast doesn't work
  // with shared libraries linked with gcc 3 and later (in GCC FAQ has bringed up
  // a way to work around the problem, but it hardly work in our case).

  std::string s=typeid(dmHello).name();
  if (s.find("Hello1Module")==std::string::npos) {
     throw std::bad_typeid();
  }
  HelloModule& helloModule = static_cast<HelloModule&>(dmHello);

  // using
  {
   std::auto_ptr<HelloInterface> h1 ( helloModule.create("xxx") );
   h1->hello();
  }

  // unloading
  DynamicModules::unload("tagHelloModule");

 }catch(const DynamicModules::Error& ex){

  // error notification
  std::cerr << "Error during loading DynamicModule" << std::endl;
  std::cerr << ex.what() << std::endl;
  return 1;
 }catch (std::bad_typeid) {
   std::cerr << "Wrong type in dynamic module" << std::endl;
 }
 return 0;
}
\end{verbatim}

\subsection{Puting Everything Together}

To obtain executable file, make following:
\begin{enumerate}

  \item Compile \verb|Hello.{so,dll}| library using command like this:
   \begin{enumerate}
    \item For UNIX with GCC compiler:
\begin{verbatim}
    g++ -shared -o Hello.so HelloModule.cpp
\end{verbatim}
    \item For Windows NT:
\begin{verbatim}
    cl Hello1Module.cpp /MD /GR /GX /D "WIN32" [...] /link -DLL /out:Hello.dll
\end{verbatim}
    (Note: /MD /GR /D "WIN32" flags must be used obviously)
   \end{enumerate}

  \item Compile Main.cpp using \verb|libDynamicModules.{a,so}| for UNIX
        or \verb|DynamicModules.lib| for Windows NT:
   \begin{enumerate}
    \item For Linux with GCC compiler:
\begin{verbatim}
    g++ Main.cpp -ldl [...] libDynamicModules.a
\end{verbatim}
    \item For Windows NT:
\begin{verbatim}
    cl Main.cpp /MD /GR /GX /D "WIN32" [...] /link DynamicModules.lib 
\end{verbatim}
   \end{enumerate}

\end{enumerate}
Executing obtained programm, you can read from standard output:
\begin{verbatim}
Hi! I'm Hello
\end{verbatim}



\section{ API DynamicModule }

\subsection{ DynamicModule Class}

 DynamicModule is abstract C++ class, and certain instance of the class
must be defined in the shared library.

 Programmer has to overload next methods of the class:
\begin{itemize}
 \item \verb|const char* name() const| - returns module name.
 \item \verb|const char* author() const| - returns module athor's name.
 \item \verb|int versionMajor() const| - returns main version number.
 \item \verb|int versionMinor() const| - returns second number version.
 \item \verb|int versionSubMinor() const| -returns third number version.
\end{itemize}
 
 Several words about version numeration: in our products we use
three-digit version number with the next  sense of digits:
\begin{itemize}
 \item[1] - main version number. It is changed if we change programmer API 
   incompatibly.
 \item[2] - second version number. It is changed if we change binary
   representation or data transmission protocol (in net case).  
 \item[3] - third version number. Third number alteration should retain 
   binary compatibility.
\end{itemize}

  Note, class sizes influence binary compatibility in shared libraries
 case, so if you added internal variable in an interface class
 you should change {\em second version number }.
 Therefore we recommend to include only abstract classes into shared 
 interface headers or to use \verb|Pimpl| idiom.
 
\subsection{ Library Tag }

 One of C++ library modules should define global variable with module type
 and unique name may being constracted such as \verb|tag<ModuleName>| or \verb|<moduleName>Tag|. 

 As you could see in above examples we write then
\begin{verbatim}
EXPORT_OBJECT(tag<ModuleName>)
\end{verbatim}

 It is syntactic sugar to conceal Windows API features (in which 
you should use special key word to export library functionality)

 One feature: the definition should be {\em outside} any 
 C++ namespace.

I. �. your file with module definition should look like underlying:
\begin{verbatim}
 ... includes ..

namespace MyCompany {

....

class MyModule: public DynamicModule
{
 ...
};
                            
} // end of namespace (MyCompany)

MyCompany::MyModule tagMyModule;


\end{verbatim}

By the way, it makes sense to add namespaces names to module
names.

I. �. it's better to do next:
\begin{verbatim}
MyCompany::MyModule tagMyCompany_MyModule;
\end{verbatim}

\subsection{ DynamicModules Class }

 DynamicModules class is singleton, which provides library management.

 Main methods are:

\begin{verbatim}
DynamicModules::load(const char* moduleName, const char* fname);
\end{verbatim}

 - loads \verb|<fname>.so| module for UNIX, or \verb|<fname>.dll| for Windows NT,
 and sets \verb|moduleName| as name for the module
 (note, for UNIX \verb|moduleName| must be the same as variable name
 having been exported via \verb|EXPORT_OBJECT| macro).

\begin{verbatim}
DynamicModules::unload(const char* moduleName);
\end{verbatim}

 - decrements internal reference counter and unloads module, if the 
 reference counter comes to zero.

 After module unloading,
 reference to the module, returned by \verb|load|, becomes invalid and
 reference use leads to the undefined behavior of the program.

\begin{verbatim}
DynamicModules::get(const char* moduleName);
\end{verbatim}

 - the same as \verb|load| except for if module is loaded already 
 reference counter doesn't change.

\section{ Error Processing } 

 DynamicModules methods can generate exception of
 \verb|DynamicModules::Error| type.

 As usually , what() method returns string - error description.
 Concrete list of possible messages depends on operational 
 system.

\section{ Final Notes} 

 The library gives facility to use weakly connected C++ components 
 model. It is more convenient than COM or XPCOM to implement
 plugin functionality in consequence of it's simplicity, on the other hand
 it doesn't make any limitations to a programmer in other component models use or
 implementing - you may combine instead of choosing.

 By the way, one of the library application fields is choice realization between 
 two technologies [in standard \verb|GradSoft| services architecture
 you may choose  underlying ��M or CORBA service, just by definition
 one of two libraries-layers in the system].

\section{ Changes List }

 \begin{itemize}
   \item[24-01-2002] \begin{enumerate}
                     \item trivial example:
                           \begin{enumerate}
                           \item placed in personal section;
                           \item simplified;
                           \item maked compilable;
                           \end{enumerate}
                     \item some misfit in API description removed.
                     \end{enumerate}
   \item[01-11-2001] created.
 \end{itemize}

\end{document}
