
\documentclass[10pt]{article}
\usepackage{graphicx}
\usepackage{verbatim}

\title{ ServiceOptions: Programming Guide 
       \newline
       \small{DocumentId:GradSoft-PR-e-02.07.2000-v1.0.3}
      }

% $Id: ProgrammingGuide_eng.tex,v 1.11 2002-05-20 18:38:45 kav Exp $

\begin{document}

\maketitle{}

\tableofcontents

\section{ Introduction }

 ServiceOptions is a component for handling of typical command options of 
 CORBA Service-like program.


 What it mean: usual CORBA Service export to "world" few objects: so called
 "Root Objects" of the service. For example, Collecion Service exports {\bf CollectionFactory } and {\bf RACollectionFactory }; NamingService exports {\bf RootNamingContext}.
 Service users obtaine access to such services via propierty mechanizme of ORB.
 
 For standart CORBA services this is call of {\em ORB::resolve\_initial\_references(name)}; but process of setting initial references in ORB with -ORBIntRef option require knowing  IOR-s of using objects during start of the program.

 Usually, "root objects" of the service are exported by
 \begin{itemize}
   \item Setting corbaloc-style IOR-s.
   \item Publishing objects IOR-s in file.
   \item Registering objects in naming service.
   \item Registering objects in IMR.
 \end{itemize}

 What do ServiceOptions: It's incapsulate all this work in one method call. I. e. you just point to serlet and name; after this you object is accessible via corbaloc-style IOR and optionally is published in other ways, depends from command line options of you server.

 ServiceOptions is based on ProgOptions\\
 (\verb|www.gradsdoft.kiev.ua/eng/Products/ToolBox/ProgOptions/ProgGuide/ProgrammingGuide_eng.html|)

 This document is an unformal description, for full specification, please, use API reference \\
 (\verb|http://www.gradsoft.kiev.ua/common/ToolBox/ServiceOptions/API/index.html|).

\section{ General Description  }

 Programmer must perform the next steps:
 \begin{enumerate}
  \item Create object of type ServiceOptions.
  \item Set names for provided CORBA objects by calling \verb|ServiceOptions::putServiceName|
  \item Call \verb|ServiceOptions::parse| for parsing options (options
        \verb|--help| and \verb|--config <filename>|, if exist, will be handled,
        see ProgOptions ProgrammingGuide for detail)
  \item After creating root objects call method \verb|ServiceOptions::bindServiceObject|
 \end{enumerate}
 In result, root object references will be accessible via corbaloc-style URL
 (corbaloc::host:port/Name) and program will understood next options:
\begin{itemize}
 \item \verb|--with-naming| - initial objects are registered in NameService
 \item \verb|--ior-stdout| - stringifized object references are printed on standart output of a program.
 \item \verb|--ior-file-<name>| - IOR of object wiht name \verb|<name>| will be published to file with name \verb|<argument of thid option>|
\end{itemize}
in addition to common ORB options.
  
\section{ Trivial example  }
  Trivial example which illustrate ServiceOptions usage is follow:

 \paragraph{A.} Once the following IDL-interface exist:

\begin{verbatim}
interface HelloWorlder
{
  void hello_world();
};
\end{verbatim}

  \paragraph{B.} Then we can write next programm:

\begin{verbatim}

#include <tao/CORBA.h>                          //
#include <tao/PortableServer/PortableServer.h>  // for TAO-1.2
#include <orbsvcs/CosNamingC.h>                 //
#include <HelloWorlderS.h>                      //

#include <iostream>
using namespace std;

#include <GradSoft/ServiceOptions.h>  // do'nt forget
using namespace GradSoft;             //

class HelloWorlder_impl:public POA_HelloWorlder           // interface
{                                                         // implementation
public:                                                   //
  void hello_world() { cout << "Hello, world" << endl; }  //
};                                                        //

int main(int argc, char** argv)
{
 ServiceOptions options;
 options.putServiceName("HelloService");   // set service name
 if (!options.parse(argc,argv)) return 1;  // parse options set

// use copy of ServiceOptions's internal (argument count,argument vector) pair
// to make it possible
// ORB_init() takes comand-line options in ProgOptions config file:

 ProgOptions::ArgsHolder argsHolder;
 argsHolder.takeArgv(options);
 CORBA::ORB_var orb = CORBA::ORB_init(argsHolder.argc,argsHolder.argv);

 CORBA::Object_var poaObj =                                             // standard
     orb->resolve_initial_references("RootPOA");                        // steps
 PortableServer::POA_var poa = PortableServer::POA::_narrow(poaObj);    //
 PortableServer::POAManager_var poaManager = poa->the_POAManager();     //
 poaManager->activate();                                                //

 HelloWorlder_impl helloWorlder_impl;   // create object

 HelloWorlder_var helloWorlder = helloWorlder_impl._this();

 options.bindServiceObject(     // principal call
    orb,
    helloWorlder,
    &helloWorlder_impl,
    "HelloService",
    true
 );

 orb->run();
 orb->destroy();
 return 0;
}

\end{verbatim}

 This program handle next options:
 \begin{enumerate}
 \item \verb|--with-naming|, \verb|--ior-stdout|, \verb|--ior-file-myName|
       as it was described above;
 \item ����� \verb|--help| � \verb|--config|
       as it was described in Programming guide for ProgOptions package;
 \item current ORB options.
 \end{enumerate}


\section{ Programming Environment Conventions  }

\begin{enumerate}
 \item ProgOptions Programming Environment Conventions must be set. (see ProgOptions Programming Guide)
 \item Using ServiceOptions with TAO-1.2,
       you must to include header file PortableServer.h before including of ServiceOptions.h.
\end{enumerate}

\section{Changes}

 \begin{itemize}
    \item[06.02.2002] - example modified: run ORB\_init with copy of argument vector
                        obtained by using getArgsHolder() is accepted
   \item [17.01.2002] - example verified
   \item [03.01.2002] - touch before 1.4.0 public release.
   \item [18.02.2001] - review, added formal document attributes.
   \item [02.07.2000] - initial revision.
 \end{itemize}

\end{document}
